\documentclass[xcolor={usenames,svgnames,x11names,table}]{beamer}

\usetheme{SBU}

\usepackage{mypackages}
\usepackage{mycommands}
%\usepackage{charter} % for Charter font

\title[TSL scope]{%
    \texorpdfstring{%
    Capturing scope ambiguity with \\ Tier-Local Syntax}
    {Capturing scope ambiguity with Tier-Local Syntax}
    }
\author[Liu]{\texorpdfstring{%
                            Lei Liu
                            }
                            {Lei Liu} 
                        }
\institute{Stony Brook University\\\texttt{lei.liu.1@stonybrook.edu}}
\date{CompPhon workshop\\Dec 12, 2016}


\begin{document}

\unnumbered{
\begin{frame}
	\titlepage
\end{frame}
}

\begin{frame}{Outline}
    \tableofcontents
\end{frame}

\section[Puzzle]{Puzzle: C-Command vs. TSL syntax}
\subsection{C-Command}
\begin{frame}{C-Command at work}
\begin{tabularx}{\textwidth}{r|c|c}
&Syntax&TSL\\
C-Command&$\color{teal}{\checkmark}$&\color{Brown}\shrug\\
\end{tabularx}

\begin{example}<2->
    \begin{exe}
        \ex<2-> Someone read every book. \hfill \surface, \inverse
    \end{exe}
\end{example}
    \begin{itemize}
    	\item<3-> The scope of $\alpha$ $=$  the C-Command domain of $\alpha$. \citep{may1985logical}
    \end{itemize}
\begin{multicols}{2}
\onslide<3->
\begin{tabularx}{.95\linewidth}{X|X}
\surface&\inverse\\
\hline
$\text{Someone}_{i}$ $\mathbb{C}$ $\text{every.book}_j$&$\text{every.book}_j$ $\mathbb{C}$ \par $\text{Someone}_{i}$\\
\end{tabularx}
\onslide<4->
 	\begin{forest}
        	[S
            	[$\text{every.book}_j$]
            	[S
                	[$\text{Someone}_{i}$ read $\text{t}_j$, roof]
                ]
            ]
    \end{forest}

\end{multicols}
\end{frame}

\subsection{TSL Syntax}
\begin{frame}{TSL syntax}
    \begin{itemize}
    	\item Dependencies of syntax captured by Tier-based Strictly Local (TSL) grammars over trees \citep{Graf16Yaletalk}
    	\item $\mathbb{T} = \{s, QPs\}$
    \end{itemize}
    \begin{forest}
    [,phantom
    [S, name=s
        [Someone, name=np1]
        [VP
            [read]
            [every.book, name=np2]
        ]
    ]
    [s,name=ts, color = {DarkCyan}
    	[someone, name=tnp1, color = {DarkCyan}]
    	[every.book, name=tnp2, color = {DarkCyan}]
    ]
    ]
    %
    \foreach \Node in {s,np1,np2}
       \draw[->,dashed,teal,bend left=15] (\Node) to (t\Node);
\end{forest}

\pause
On the tier...
    \begin{itemize}
    	\item {\color{Brown}C-Command}  gone
    	\item {\color{Teal}Locality} gained
    \end{itemize}
\pause
Can TSL handle scope interpretation without C-Command? Yes!
\end{frame}

\section[Proposal]{Proposal: TSL with proper C-Command domain}
\begin{frame}{Proposal}
For a quantificational phrase:
    \begin{itemize}
        \item Higher on tier, higher in scope
        \item Ambiguous when mutual C-Command found in... 
            \begin{itemize}
                \item declarative sentences, within a {\color{Brown}TP}
                \item \emph{wh}-questions, within a {\color{Brown}vP}
            \end{itemize}
    \end{itemize}
\end{frame}

\begin{frame}{Declarative sentence}

\begin{forest}
   [,phantom
       [CP,name=c[TP,name=s
            [Someone, name=np1]
            [T'
                [T]
                [VP
                    [read]
                    [{every book}, name=np2]
                ]
            ]
       ]]
       [CP,name=tc,color = {DarkCyan}[TP,name=ts, color = {Brown}
           [someone,name=tnp1, color = {DarkCyan}]
           [{every book},name=tnp2, color = {DarkCyan}]
       ]]
   ]
   %
   \foreach \Node in {c,s,np1,np2}
       \draw[->,dashed,teal,bend left=15] (\Node) to (t\Node);
\end{forest}
\pause
\begin{center}
{\color{Brown}Ambiguous!}
\end{center}
\end{frame}

\begin{frame}{\emph{Wh}-questions}

\begin{forest}
   [,phantom
       [CP,name=c
       		[Who, name=np1]
       		[TP
            	[T'
                	[T]
                	[vP,name=v
                    	[read]
                    	[{every book}, name=np2]
                	]
            	]
       		]	
       ]
       [cp,name=tc,color = {DarkCyan}
       		[who,name=tnp1, color = {DarkCyan}]
           	[vP,name=tv, color = {Brown}
           		[every.book, name=tnp2, color = {DarkCyan}]
       		]
   		]
   	]
   %
   \foreach \Node in {c,v,np1,np2}
       \draw[->,dashed,teal,bend left=15] (\Node) to (t\Node);
\end{forest}
\pause
\begin{exe}
	\ex Who read every book? \hfill {\color{Brown}\emph{who}~\before~\emph{every.book}!}
\end{exe}
\end{frame}

\section[Prediction]{Prediction: wh-in-situ and QP-domain correlation}
\begin{frame}{Prediction}
    \begin{itemize}[<+->]
        \item subject \emph{wh}-questions are ambiguous in \emph{wh}-in-situ languages
        \begin{exe}
        	\ex \gll
        	Shenme.she yao.LE mei.wei xiangdao?\\
        	what.snake bite.LE every.CL guide\\
        	``What snake bit every guide?'' \hfill \surface, \inverse
        \end{exe}
        \item more ``complex'' the QPs, smaller the domain relevant for C-Command evaluation.
        	\begin{itemize}
        		\item QP - TP
        		\item wh, QP - vP
        		\item double objects - smaller than vP
        	\end{itemize}
    \end{itemize}
\end{frame}

\unnumbered{
\begin{frame}{Chinese subject \emph{wh}-question}
\begin{forest}
   [,phantom
       [CP,name=c[TP,name=s
            [{Shenme.she\\\emph{what.snake}}, name=np1]
            [T'
                [T]
                [VP
                    [{yao.LE\\\emph{bite.LE}}]
                    [{mei.wei.xiangdao\\\emph{every.CL.guide}}, name=np2]
                ]
            ]
       ]]
       [CP,name=tc,color = {DarkCyan}[TP,name=ts, color = {Brown}
           [{Shenme.she\\\emph{what.snake}},name=tnp1, color = {DarkCyan}]
           [{mei.wei.xiangdao\\\emph{every.CL.guide}},name=tnp2, color = {DarkCyan}]
       ]]
   ]
   %
   \foreach \Node in {c,s,np1,np2}
       \draw[->,dashed,teal,bend left=15] (\Node) to (t\Node);
\end{forest}
\begin{center}
{\color{Brown}Ambiguous!}
\end{center}
\end{frame}
}

\unnumbered{
\begin{frame}{Double Object Construction}
\begin{multicols}{2}
\begin{forest}
   [,phantom
       [CP,name=c
       		[vP, roof
            [John]
            [v'
                [v]
                [vP
                    [a.student, name = np1]
                    [v'
	                    [v]
	                    [vP, name = v
	                    	[give]
	                    	[every.book, name = np2]
	                    ]
	                ]
                ]
            ]
           	]
       ]
       [CP,name=tc,color = {DarkCyan}
       		[a.student,name=tnp1, color = {DarkCyan}]
       		[vP,name=tv, color = {Brown}
           	   	[every.book,name=tnp2, color = {DarkCyan}]
       		]
       ]
   ]
   %
   \foreach \Node in {c,v,np1,np2}
       \draw[->,dashed,teal,bend left=15] (\Node) to (t\Node);
\end{forest}
\begin{exe}
\ex John gave a student every book.
\end{exe}
{\color{Brown}\emph{a.student}~\before~\emph{every.book}!}
\end{multicols}
\end{frame}
}

\unnumbered{
\begin{frame}{Reference}
\bibliography{Syntaxbib2016fall}
\end{frame}
}
\end{document}
