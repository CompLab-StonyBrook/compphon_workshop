\documentclass{beamer}
\usepackage{graphicx}
\graphicspath{ {images/} }

\usepackage{tikz}
\usepackage{multicol}
\usetikzlibrary{decorations.pathmorphing}
\usepackage{soul}
\usepackage{tipa}
\usepackage{tcolorbox}
\usepackage{amssymb}
\usepackage{pifont}

\mode<presentation> {
\usetheme{Frankfurt}
\usecolortheme{dolphin}
\setbeamertemplate{navigation symbols}{} % remove the navigation symbols
%\setbeamertemplate{footline}[frame number]
% empty footline except page number in bottom right
\setbeamerfont{footline}{size=\scriptsize}
\newcommand{\numberedfootline}{%
    \setbeamertemplate{footline}{\color{blue!50}\hfill\insertframenumber\hspace{0.5em}\smallskip}
}
\numberedfootline
}

\usepackage{graphicx} % Allows including images
\usepackage{booktabs} % Allows the use of \toprule, \midrule and \bottomrule in tables

\newenvironment<>{varblock}[2][.9\textwidth]{%
  \setlength{\textwidth}{#1}
  \begin{actionenv}#3%
    \def\insertblocktitle{#2}%
    \par%
    \usebeamertemplate{block begin}}
  {\par%
    \usebeamertemplate{block end}%
  \end{actionenv}}



% ====================================================
%	TITLE PAGE
% ====================================================

\title[%
	To Achieve Harmony We Only Need One Tier]{%
	To Achieve Harmony We Only Need One Tier} 
\author{Al\"ena Aks\"enova}
\institute[Stony Brook University]
{ \vspace{-1em}
Stony Brook University \\
\vspace{6em}
}
\date
{Computational Phonology Workshop \\ \medskip
{\small Stony Brook University \\
December 12, 2016} }

\begin{document}
\setbeamertemplate{enumerate items}[default] % uncomment to get ugly enumerate

% commands
\newcommand{\highlight}[1]{\textbf{\textcolor{blue}{#1}}}
\newcommand{\imp}[1]{\textbf{#1}}
\newcommand{\citecolor}[1]{{\color{gray!85}{#1}}}
\newcommand{\blue}[1]{{\color{blue}{#1}}}
\newcommand{\linecell}[2][l]{% next line in a cell
\begin{tabular}[#1]{@{}l@{}}#2\end{tabular}}
%\newcommand{\comment}[1]{}
\newcommand\unnumbered{\setbeamertemplate{footline}{}\addtocounter{framenumber}{-1}}

%------------------------------------------------

{\unnumbered
\begin{frame}
\titlepage
\end{frame}
}





% ====================================================
%	MY GOAL
% ====================================================

\begin{frame}
\frametitle{My goal}
\begin{itemize}
	\item Several TSL grammars are needed to capture several long-distance phonological dependencies.
	\medskip
	\item For a single vowel harmony, only one tier is enough.
	\item There are \textcolor{blue!60!black}{double vowel harmonies}, where primary and secondary features are being spread.
	\item Do we need more than one tier to model such harmonies?
\end{itemize}
\pause
\bigskip
\begin{center}
\textcolor{blue!75!black!50!white}{\textbf{\Large One tier is always enough!}}
\end{center}
\end{frame}

%------------------------------------------------

{\unnumbered
\begin{frame}
\frametitle{Outline}
\tableofcontents
\end{frame}
}





% ====================================================
%	TSL PHONOLOGY
% ====================================================
\section[TSL phonology]{TSL phonology}
\subsection*{TSL}

%\begin{frame}
%\frametitle{SL phonology}
%{\large \textbf{Strictly Local grammars} capture local dependencies.}
%\bigskip
%
%\pause
%
%\begin{example}[Intervocalic \emph{`s'} voicing in German]
%		\begin{itemize}
%			\item \textcolor{blue!60!black}{Single \emph{`s'} becomes voiced in-between two vowels:} \\
%				\begin{tabular}{lcll}
%					F\textcolor{green!50!black}{ase}r & $\rightarrow$ & fa[z]er & `fiber'  \\
%					re\textcolor{green!50!black}{ise}n & $\rightarrow$ & rei[z]en & `to.travel'
%				\end{tabular}
%			\pause
%			\item \textcolor{blue!60!black}{Other consonants are unaffected:} \\
%				\begin{tabular}{lcll}
%					Wasser & $\rightarrow$ & wa[ss]er & `water'  \\
%					reiste & $\rightarrow$ & rei[s]te & `traveled'
%				\end{tabular}
%			\pause\medskip
%			\item $G_{SL}$ = \{*VsV\}
%			\pause\medskip
%			\item \textcolor{red!75!black}{*}re\textcolor{red!75!black}{i[s]e}n \qquad
%				$^{\textcolor{green!50!black}{ok}}$rei[z]en \qquad
%				$^{\textcolor{green!50!black}{ok}}$rei[st]e
%		\end{itemize}
%	\end{example}
%\end{frame}

%------------------------------------------------

\begin{frame}
\frametitle{TSL phonology}
{\large \textbf{Tier-based Strictly Local grammars} capture non-local dependencies in a local fashion by projecting elements of a certain type on a \emph{tier}.}
\bigskip
%
\pause
%
%\begin{example}[\emph{-\=alis}/\emph{-\=aris} alternation in Latin]
%\begin{itemize}
%	\item \textcolor{blue!60!black}{\emph{-\=alis} is chosen when the previous liquid is \emph{`r'}:} \\
%		\begin{tabular}{llcll}
%			\textcolor{green!50!black}{l}\={\textsci}b\textcolor{green!50!black}{r}a & `pound'
%			& $\rightarrow$ &
%			\textcolor{green!50!black}{l}\={\textsci}b\textcolor{green!50!black}{r}\={a}\textcolor{green!50!black}{l}is & `weighing one pound'
%		\end{tabular}
		%
%	\item \textcolor{blue!60!black}{\emph{-\=aris} is chosen when the previous liquid is \emph{`l'}:} \\
%		\begin{tabular}{llcll}
%			m\={\textsci}\textcolor{green!50!black}{l}es & `soldier'
%			& $\rightarrow$ &
%			m\={\textsci}\textcolor{green!50!black}{l}it\=a\textcolor{green!50!black}{r}is & `military'
%		\end{tabular}
	%
%	\pause
	%
%	\item Not SL: there can be unbounded amount of material in-between the two liquids.
	%
%	\pause
	%
%	\item $T$ = \{l, r\}, $G_{TSL}$ = \{*ll, *rr\}
%\end{itemize}
%\end{example}
%\end{frame}


%------------------------------------------------

%\begin{frame}
%\frametitle{TSL phonology [cont.]}
%\begin{example}[\emph{-\=alis}/\emph{-\=aris} alternation in Latin]
%\begin{itemize}
%	\item $T$ = \{l, r\}, $G_{TSL}$ = \{*ll, *rr\}
%\end{itemize}
%
%\pause
%
%\begin{center}
%\begin{tikzpicture}
%\node (1) at (0.5,-0.01) {l};
%\node (2) at (1,-0.02) {\={\textsci}};
%\node (3) at (1.5,0) {b};
%\node (4) at (2,-0.05) {r};
%\node (5) at (2.5,-0.02) {\={a}};
%\node (6) at (3,-0.01) {l};
%\node (7) at (3.5,-0.01) {i};
%\node (8) at (4,-0.05) {s};
%
%\node (01) at (0.5,1) {l};
%\node (04) at (2,0.95) {r};
%\node (06) at (3,1) {l};
%
%\foreach \Source/\Target in {%
%        1.north/01.south,
%        4.north/04.south,
%        6.north/06.south%
%    }
%\draw (\Source) to (\Target);
%
%\draw[dotted] (0.25,0.5) to (5,0.5);
%\node at (4.5,0.65) {{\tiny liquids}};
%
%\node at (0.3,1.5) {$^{\textcolor{green!50!black}{ok}}$l\={\textsci}br\={a}lis};
%\end{tikzpicture}
%\end{center}
%
%\pause
%
%\begin{center}
%\begin{tikzpicture}
%\node (1) at (0.5,-0.01) {l};
%\node (2) at (1,-0.02) {\={\textsci}};
%\node (3) at (1.5,0) {b};
%\node (4) at (2,-0.05) {r};
%\node (5) at (2.5,-0.02) {\={a}};
%\node (6) at (3,-0.05) {r};
%\node (7) at (3.5,-0.01) {i};
%\node (8) at (4,-0.05) {s};
%
%\node (01) at (0.5,1) {l};
%\node (04) at (2,0.95) {\alert{r}};
%\node (06) at (3,0.95) {\alert{r}};
%
%\foreach \Source/\Target in {%
%        1.north/01.south,
%        4.north/04.south,
%        6.north/06.south%
%    }
%\draw (\Source) to (\Target);
%
%\draw[dotted] (0.25,0.5) to (5,0.5);
%\node at (4.5,0.65) {{\tiny liquids}};
%
%\node at (0.3,1.5) {\alert{*}l\={\textsci}br\={a}ris};
%\end{tikzpicture}
%\end{center}
%\end{example}
%\end{frame}



%\section[Single harmony]{Single harmony}
%\subsection*{Single}

%\begin{frame}
%\frametitle{Single harmony}
\begin{itemize}
	\item \textcolor{blue!60!black}{\emph{Single harmony} is a harmony where only one feature is being transmitted among certain set of vowels.} \\ \smallskip
		{\small 
		\textsc{Finnish, Hungarian}: front/back harmony \\
		\textsc{Lokaa}: ATR harmony among non-high vowels \\
		\textsc{Kingala}: height harmony}
	%
	\pause\medskip
	%
	\item Harmonizing elements and blockers are projected on a tier.
	\item Transparent elements are ignored.
\end{itemize}
\end{frame}

%------------------------------------------------

\begin{frame}
\frametitle{Harmony in Lokaa}
\begin{itemize}
	\item Harmonizing elements and blockers are projected on a tier.
	\item Transparent elements are ignored.
\end{itemize}
%
\pause
%
\begin{example}[Vowel harmony in \textsc{Lokaa}]
\begin{itemize}
	\item \textcolor{blue!60!black}{Non-high vowels harmonize in ATR.} \textcolor{gray}{(Akinlabi 2009)} \\
		\begin{tabular}{rllll}
			tense: & \hspace{1em} \textcolor{green!50!black}{\`e}s\`is\textcolor{green!50!black}{\`o}n & `smoke' &
			 l\textcolor{green!50!black}{\`e}j\`im\textcolor{green!50!black}{\`\textschwa} & `matriclan' \\
			lax: &\hspace{1em} \textcolor{green!50!black}{\`\textepsilon}s\'is\textcolor{green!50!black}{\`\textopeno}n & `housefly' \hspace{1em} & \textcolor{green!50!black}{\'\textopeno}t\'u:m\textcolor{green!50!black}{\'a} & `need'
		\end{tabular}
	%
	\medskip
	\pause
	%
	\item $T$ = \{e, o, \textschwa, \textepsilon, \textopeno, a\}, $H_{ATR}$ = \{*e\textepsilon, *e\textopeno, *ea, *o\textopeno, *\textepsilon o, ...\}
\end{itemize}
%
\pause
%
\vspace{-0.8em}
\begin{figure}
\begin{tikzpicture}
\node (1) at (0.5,0.05) {\`\textepsilon};
\node (2) at (1,0) {s};
\node (3) at (1.5,0.03) {\`i};
\node (4) at (2,0) {s};
\node (5) at (2.5,0.05) {\`\textopeno};
\node (6) at (3,0) {n};
%
\node (01) at (0.5,1) {\textepsilon};
\node (05) at (2.5,1) {\textopeno};
%
\foreach \Source/\Target in {%
	1.north/01.south,
	5.north/05.south%
    }
\draw (\Source) to (\Target);
%
\draw[dotted] (0.25,0.5) to (4.2,0.5);
\node at (3.5,0.65) {{\tiny [-hi] vowels}};
%
\node at (0.8,1.5) {$^{\textcolor{green!50!black}{ok}}$\`\textepsilon s\`is\`\textopeno n};
\end{tikzpicture}
%
\hspace{1.3em}
\begin{tikzpicture}
\node (1) at (0.5,0.05) {\`\textepsilon};
\node (2) at (1,0) {s};
\node (3) at (1.5,0.03) {\`i};
\node (4) at (2,0) {s};
\node (5) at (2.5,0.05) {\`o};
\node (6) at (3,0) {n};
%
\node (01) at (0.5,1) {\alert{\textepsilon}};
\node (05) at (2.5,1) {\alert{o}};
%
\foreach \Source/\Target in {%
	1.north/01.south,
	5.north/05.south%
    }
\draw (\Source) to (\Target);
%
\draw[dotted] (0.25,0.5) to (4.2,0.5);
\node at (3.5,0.65) {{\tiny [-hi] vowels}};
%
\node at (0.8,1.5) {\textcolor{red}{*}\`\textepsilon s\`is\`on};
\end{tikzpicture}
\end{figure}
\end{example}
\end{frame}

%------------------------------------------------

%\begin{frame}
%\frametitle{Single harmony: Assamese}
%\begin{itemize}
%	\item Harmonizing elements and blockers are projected on a tier
%	\item Transparent elements are ignored
%\end{itemize}
%
%\pause
%
%\begin{example}[Vowel harmony in \textsc{Assamese}]
%\begin{itemize}
%	\item \textcolor{blue!60!black}{High vowels spread [+ATR] regressively, [\textscripta] is a blocker} \textcolor{gray}{(Mahanta 2008)} \\
%	\begin{tabular}{llcll}
%	z\textcolor{green!50!black}{\textupsilon}r & `strength' & $\rightarrow$ & z\textcolor{green!50!black}{u}r-\textcolor{green!50!black}{i} & `strong' \\
%	x\textcolor{green!50!black}{\textupsilon}n & `gold' & $\rightarrow$ & x\textcolor{green!50!black}{\textupsilon}n-\textcolor{red!60!black}{\textscripta}l\textcolor{green!50!black}{i} & `golden'
%	\end{tabular}
	%
%	\medskip
%	\pause
	%
%	\item $T$ = \{\textupsilon, u, \textscripta, i, ...\}, $H_{ATR}$ = \{*\textupsilon i, *\textupsilon u,  *u\textscripta, *i\textscripta...\}
%\end{itemize}
%
%\pause
%
%\vspace{-0.8em}
%
%\begin{figure}
%
%\begin{tikzpicture}
%\node (1) at (0.5,0) {x};
%\node (2) at (1,0) {\textupsilon};
%\node (3) at (1.5,0) {n};
%\node (4) at (2,0) {\textscripta};
%\node (5) at (2.5,0.03) {l};
%\node (6) at (3,0.03) {i};
%
%\node (02) at (1,1) {\textupsilon};
%\node (04) at (2,1) {\textscripta};
%\node (06) at (3,1.03) {i};
%
%\foreach \Source/\Target in {%
%	2.north/02.south,
%	4.north/04.south,
%	6.north/06.south%
%    }
%\draw (\Source) to (\Target);
%
%\draw[dotted] (0.25,0.5) to (4.2,0.5);
%\node at (3.65,0.65) {{\tiny harmony}};
%
%\node at (0.8,1.5) {$^{\textcolor{green!50!black}{ok}}$x\textupsilon n\textscripta li};
%\end{tikzpicture}
%
%\hspace{1em}
%
%\begin{tikzpicture}
%\node (1) at (0.5,0) {x};
%\node (2) at (1,0) {u};
%\node (3) at (1.5,0) {n};
%\node (4) at (2,0) {\textscripta};
%\node (5) at (2.5,0.03) {l};
%\node (6) at (3,0.03) {i};
%
%\node (02) at (1,1) {\alert{u}};
%\node (04) at (2,1) {\alert{\textscripta}};
%\node (06) at (3,1.03) {i};
%
%\foreach \Source/\Target in {%
%	2.north/02.south,
%	4.north/04.south,
%	6.north/06.south%
%    }
%\draw (\Source) to (\Target);
%
%\draw[dotted] (0.25,0.5) to (4.2,0.5);
%\node at (3.65,0.65) {{\tiny harmony}};
%
%\node at (0.8,1.5) {\textcolor{red}{*}xun\textscripta li};
%\end{tikzpicture}
%
%\end{figure}
%\end{example}
%\end{frame}


% ====================================================
%	SECONDARY HARMONY
% ====================================================
\section[Double harmony]{Secondary harmony}
\subsection*{Double}

\begin{frame}
\frametitle{Secondary harmony}
\begin{itemize}
	\item In numerous languages we observe \textcolor{blue!60!black}{\emph{double harmony,} 
		where two features are being transmitted}. \\ \smallskip
		{\small 
		\textsc{Turkish, Yakut, Kirghiz}: front/back and rounding \\
		\textsc{Buryat, Mongolian}: ATR and rounding \\
		\textsc{Coeur d'Alene}: ATR and height }
	\pause \bigskip
	\item We expect to need a separate tier for secondary harmonies.
	\item Is it true?
\end{itemize}
\end{frame}

%------------------------------------------------

\begin{frame}
\frametitle{Secondary harmony: Buryat}
\begin{itemize} { \small
	\item \textcolor{blue!60!black}{Primary:} all vowels harmonize in ATR.
	\item \textcolor{blue!60!black}{Secondary:} non-high vowels agree in rounding, high ones block it. }
%	\item The harmony is root-controlled.
\end{itemize}
%
\pause
%
\begin{example}[Double vowel harmony in \textsc{Buryat}]
\begin{itemize}
	\item \textbf{Roots:} \,\,\,\textopeno r `enter' \qquad to:r `wander'
	\item \textbf{Affixes:} -\textupsilon:l, -u:l  \qquad\qquad\,\,\,\,\,\,\,\, `\textsc{caus}' \quad (V:$^{+hi}$l) \\
		\qquad\,\,\,\,\,\,\,\,\, -a:d, -\textopeno:d, -e:d, -o:d \,\,\, `\textsc{perf}' \quad (V:$^{-hi}$d) \\
		\pause\medskip
		\begin{tabular}{lll}
\multicolumn{3}{l}{\footnotesize \textcolor{red!75!yellow!80!white}{ \# lax word; two adjacent non-high vowels harmonize \vspace{-0.2em}}} \\
\textcolor{green!50!black}{\textopeno}r-\textcolor{green!50!black}{\textopeno:}d & `enter-\textsc{perf}' & *\textopeno r-a:d \pause \\
\multicolumn{3}{l}{\footnotesize \textcolor{red!75!yellow!80!white}{ \# lax word; high vowel blocks labial harmony \vspace{-0.2em}}} \\
\textcolor{green!50!black}{\textopeno}r-\textcolor{green!50!black}{\textupsilon:}l-\textcolor{green!50!black}{a:}d & `enter-\textsc{caus-perf}' & *\textopeno r-\textupsilon:l-\textopeno:d \pause \\
\multicolumn{3}{l}{\footnotesize \textcolor{red!75!yellow!80!white}{ \# tensed word; two adjacent non-high vowels harmonize \vspace{-0.2em}}} \\
t\textcolor{green!50!black}{o:}r-\textcolor{green!50!black}{o:}d & `wander-\textsc{perf}' & *to:r-e:d \pause \\
\multicolumn{3}{l}{\footnotesize \textcolor{red!75!yellow!80!white}{ \# tensed word; high vowel blocks labial harmony \vspace{-0.2em}}} \\
t\textcolor{green!50!black}{o:}r-\textcolor{green!50!black}{u:}l-\textcolor{green!50!black}{e:}d & `wander-\textsc{caus-perf}' & *to:r-u:l-o:d
\end{tabular}
\end{itemize}
\vspace{-0.7em}
\end{example}
\end{frame}

%------------------------------------------------

\begin{frame}
\frametitle{Secondary harmony: Buryat [cont.]}
\begin{itemize}
	\item Tier $T$ = \{a, e, \textopeno, o, \textupsilon, u\}
	\pause\bigskip
	%
	\item \textcolor{blue!60!black}{Primary:} all vowels harmonize in ATR.
	\item No *V$_{\alpha ATR}$V$_{-\alpha ATR}$ bigrams \\
		{\small $H_{ATR}$\,\,= \{*\textopeno o, *o\textopeno, *\textupsilon u, *u\textupsilon, *ae, *ea, 
			*\textupsilon o, *o\textupsilon, *\textupsilon e, *e\textupsilon, *\textopeno u, \\ \qquad\qquad
			\quad *u\textopeno, *\textopeno e, *e\textopeno, *oa, *ao, *au, *ua\} }
	\pause\bigskip
	%
	\item \textcolor{blue!60!black}{Secondary:} non-high vowels harmonize in rounding, high vowels are blockers: rounded non-high vowels can either occur in the first syllable, or be licensed by harmony.
	\item No *V$^{-hi}_{\alpha round}$V$^{-hi}_{-\alpha round}$ and *V$^{+hi}$V$^{-hi}_{+round}$ bigrams \\
		{\small $H_{round}$\,\,= \{*\textopeno a, *a\textopeno, *eo, *oe, *\textupsilon\textopeno, *uo\} }
\end{itemize}
\end{frame}

%------------------------------------------------

\begin{frame}
\frametitle{Secondary harmony: Buryat [cont.]}
\begin{itemize}
	\item $T$ = \{a, e, \textopeno, o, \textupsilon, u\}
	\item {\small $H_{ATR}$\,\,= \{*\textopeno o, *o\textopeno, *\textupsilon u, *u\textupsilon, *ae, *ea, 
			*\textupsilon o, *o\textupsilon, *\textupsilon e, *e\textupsilon, *\textopeno u, \\ \qquad\qquad
			\quad *u\textopeno, *\textopeno e, *e\textopeno, *oa, *ao, *au, *ua\} }
	\item {\small $H_{round}$\,\,= \{*\textopeno a, *a\textopeno, *eo, *oe, *\textupsilon\textopeno, *uo\} }
\end{itemize}
\pause
\begin{example}[Double vowel harmony in \textsc{Buryat}]
\begin{tabular}{lll}
\hspace{2em} t\textcolor{green!50!black}{o:}r-\textcolor{green!50!black}{o:}d & `wander-\textsc{perf}' & *to:r-e:d
\end{tabular}
\pause
\begin{figure}
\begin{tikzpicture}
\node (0) at (0,0.027) {t};
\node (1) at (0.5,0) {o:};
\node (2) at (1,0) {r};
\node (3) at (1.5,0) {o:};
\node (4) at (2,0.04) {d};
%
\node (01) at (0.5,1) {o:};
\node (03) at (1.5,1) {o:};
%
\foreach \Source/\Target in {%
        1.north/01.south,
	3.north/03.south%
    }
\draw (\Source) to (\Target);
%
\draw[dotted] (-0.25,0.5) to (3,0.5);
\node at (2.5,0.65) {{\tiny harmony}};
%
\node at (0.3,1.5) {$^{\textcolor{green!50!black}{ok}}$to:ro:d};
\end{tikzpicture}
%
\hspace{1.5em}
%
\begin{tikzpicture}
\node (0) at (0,0.027) {t};
\node (1) at (0.5,0) {o:};
\node (2) at (1,0) {r};
\node (3) at (1.5,0) {e:};
\node (4) at (2,0.04) {d};
%
\node (01) at (0.5,1) {\alert{o:}};
\node (03) at (1.5,1) {\alert{e:}};
%
\foreach \Source/\Target in {%
        1.north/01.south,
	3.north/03.south%
    }
\draw (\Source) to (\Target);
%
\draw[dotted] (-0.25,0.5) to (3,0.5);
\node at (2.5,0.65) {{\tiny harmony}};
%
\node at (0.3,1.5) {\textcolor{red}{*}to:re:d};
\end{tikzpicture}
\end{figure}
\end{example}
\end{frame}

%------------------------------------------------

\begin{frame}
\frametitle{Secondary harmony: Buryat [cont.]}
\begin{itemize}
	\item $T$ = \{a, e, \textopeno, o, \textupsilon, u\}
	\item {\small $H_{ATR}$\,\,= \{*\textopeno o, *o\textopeno, *\textupsilon u, *u\textupsilon, *ae, *ea, 
			*\textupsilon o, *o\textupsilon, *\textupsilon e, *e\textupsilon, *\textopeno u, \\ \qquad\qquad
			\quad *u\textopeno, *\textopeno e, *e\textopeno, *oa, *ao, *au, *ua\} }
	\item {\small $H_{round}$\,\,= \{*\textopeno a, *a\textopeno, *eo, *oe, *\textupsilon\textopeno, *uo\} }
\end{itemize}
%
\begin{example}[Double vowel harmony in \textsc{Buryat}]
\begin{tabular}{lll}
\hspace{2em} t\textcolor{green!50!black}{o:}r-\textcolor{green!50!black}{u:}l-\textcolor{green!50!black}{e:}d & `wander-\textsc{caus-perf}' & *to:r-u:l-o:d
\end{tabular}
\pause
\begin{figure}
\begin{tikzpicture}
\node (0) at (0,0.027) {t};
\node (1) at (0.5,0) {o:};
\node (2) at (1,0) {r};
\node (3) at (1.5,0) {u:};
\node (4) at (2,0.04) {l};
\node (5) at (2.5,0) {e:};
\node (6) at (3,0.04) {d};
%
\node (01) at (0.5,1) {o:};
\node (03) at (1.5,1) {u:};
\node (05) at (2.5,1) {e:};
%
\foreach \Source/\Target in {%
        1.north/01.south,
        3.north/03.south,
	5.north/05.south%
    }
\draw (\Source) to (\Target);
%
\draw[dotted] (-0.25,0.5) to (4,0.5);
\node at (3.5,0.65) {{\tiny harmony}};
%
\node at (0.4,1.5) {$^{\textcolor{green!50!black}{ok}}$to:ru:le:d};
\end{tikzpicture}
%
\hspace{1em}
%
\begin{tikzpicture}
\node (0) at (0,0.027) {t};
\node (1) at (0.5,0) {o:};
\node (2) at (1,0) {r};
\node (3) at (1.5,0) {u:};
\node (4) at (2,0.04) {l};
\node (5) at (2.5,0) {o:};
\node (6) at (3,0.04) {d};
%
\node (01) at (0.5,1) {o:};
\node (03) at (1.5,1) {\alert{u:}};
\node (05) at (2.5,1) {\alert{o:}};
%
\foreach \Source/\Target in {%
        1.north/01.south,
        3.north/03.south,
	5.north/05.south%
    }
\draw (\Source) to (\Target);
%
\draw[dotted] (-0.25,0.5) to (4,0.5);
\node at (3.5,0.65) {{\tiny harmony}};
%\draw [dashed] (1.2,0.77) -- (2.7,0.77) -- (2.7,1.2) -- (1.2,1.2) -- (1.2,0.77);
%
\node at (0.4,1.5) {\textcolor{red}{*}to:ru:lo:d};
\end{tikzpicture}
\end{figure}
\end{example}
\end{frame}

%------------------------------------------------

\begin{frame}
\frametitle{Secondary harmony: Buryat [cont.]}
\begin{itemize}
	\item \textcolor{blue!60!black}{Primary:} all vowels harmonize in ATR.
	\item \textcolor{blue!60!black}{Secondary:} non-high vowels harmonize in rounding, high vowels block this spreading.
	%
	\pause\bigskip
	%
	\item Only one tier is needed:
		\begin{itemize}
			\item Transparent elements are same for both harmonies.
			\item Secondary harmony is \emph{bounded on the vowel tier}, 
				high vowels block the spreading of [round] feature.
		\end{itemize}
\end{itemize}
\end{frame}





% ====================================================
%	UNATTESTED PATTERN
% ====================================================
\section[Unattested patterns]{Unattested patterns}
\subsection*{Unattested patterns}

\begin{frame}
\frametitle{Unattested patterns}
It is possible to imagine a language where a separate tier for the secondary harmony will be needed:
\bigskip
\pause
\begin{itemize}
	\item Secondary harmony is also unbounded and affects only subset of the vowels that take part in the primary spreading.
	\item Secondary harmony involves a blocker that was not presented on the tier of the primary one.
\end{itemize}
\end{frame}

%------------------------------------------------

\begin{frame}
\frametitle{Unattested patterns: Pseudo-Buryat}
\begin{itemize}
	\item Secondary harmony is also unbounded and affects only subset of the vowels that take part in the primary spreading.
\end{itemize}
\pause
%
\begin{example}[Double vowel harmony in \textsc{Pseudo-Buryat}]
\begin{itemize}
	\item \textcolor{blue!60!black}{Primary:} all vowels harmonize in ATR
	\item \textcolor{blue!60!black}{Secondary:} all non-high vowels harmonize in rounding \\ \medskip
		\begin{tabular}{ll}
			\textcolor{green!50!black}{\textopeno}r\textcolor{green!50!black}{\textupsilon}l\textcolor{green!50!black}{\textopeno}d &  *\textopeno r\textupsilon lad \\
			t\textcolor{green!50!black}{o}r\textcolor{green!50!black}{u}l\textcolor{green!50!black}{o}d & *toruled
		\end{tabular}
	\medskip
	\pause
	\item One tier is not enough: there can be unbounded amount of high vowels in-between the two harmonizing non-high ones.
\end{itemize}
\end{example}
\end{frame}

%------------------------------------------------

\begin{frame}
\frametitle{Unattested patterns: Pseudo-Buryat [cont.]}
\begin{itemize}
	\item \textcolor{blue!60!black}{Primary:} all vowels harmonize in ATR.
	\item \textcolor{blue!60!black}{Secondary:} all non-high vowels harmonize in rounding.
\end{itemize}
\pause
%
\begin{example}[Double vowel harmony in \textsc{Pseudo-Buryat}]
\vspace{1em}
	\begin{tabular}{ll}
		\hspace{3em} t\textcolor{green!50!black}{o}r\textcolor{green!50!black}{u}l\textcolor{green!50!black}{o}d & *toruled
	\end{tabular}
\vspace{-0.5em}
%
\pause
%
\begin{figure}
\begin{tikzpicture}
\node (0) at (0,0.027) {t};
\node (1) at (0.5,0) {o};
\node (2) at (1,0) {r};
\node (3) at (1.5,0) {u};
\node (4) at (2,0.04) {l};
\node (5) at (2.5,0) {o};
\node (6) at (3,0.04) {d};
%
\node (01) at (0.5,1) {o};
\node (03) at (1.5,1) {u};
\node (05) at (2.5,1) {o};
%
\node (001) at (0.5,-1) {o};
\node (005) at (2.5,-1) {o};
%
\foreach \Source/\Target in {%
	001.north/1.south,
	005.north/5.south,
        1.north/01.south,
        3.north/03.south,
	5.north/05.south%
    }
\draw (\Source) to (\Target);
%
\draw[dotted] (-0.25,0.5) to (4.2,0.5);
\node at (3.5,0.65) {{\tiny ATR harmony}};
\draw[dotted] (-0.25,-0.55) to (4.2,-0.55);
\node at (3.5,-0.4) {{\tiny labial harmony}};
%
\node at (0.4,1.5) {$^{\textcolor{green!50!black}{ok}}$torulod};
\end{tikzpicture}
%
\hspace{1em}
%
\begin{tikzpicture}
\node (0) at (0,0.027) {t};
\node (1) at (0.5,0) {o};
\node (2) at (1,0) {r};
\node (3) at (1.5,0) {u};
\node (4) at (2,0.04) {l};
\node (5) at (2.5,0) {e};
\node (6) at (3,0.04) {d};
%
\node (01) at (0.5,1) {o};
\node (03) at (1.5,1) {u};
\node (05) at (2.5,1) {e};
%
\node (001) at (0.5,-1) {\alert{o}};
\node (005) at (2.5,-1) {\alert{e}};
%
\foreach \Source/\Target in {%
	001.north/1.south,
	005.north/5.south,
        1.north/01.south,
        3.north/03.south,
	5.north/05.south%
    }
\draw (\Source) to (\Target);
%
\draw[dotted] (-0.25,0.5) to (4.2,0.5);
\node at (3.5,0.65) {{\tiny ATR harmony}};
\draw[dotted] (-0.25,-0.55) to (4.2,-0.55);
\node at (3.5,-0.4) {{\tiny labial harmony}};
%\draw [dashed] (0.28,-0.8) -- (2.7,-0.8) -- (2.7,-1.2) -- (0.28,-1.2) -- (0.28,-0.8);
%
\node at (0.4,1.5) {\textcolor{red}{*}toruled};
\end{tikzpicture}
\end{figure}
\end{example}
\end{frame}

%------------------------------------------------




% ====================================================
%	DISCUSSION
% ====================================================
\section[Discussion]{Discussion}
\subsection*{Discussion}

\begin{frame}
\frametitle{Conclusion}
\textbf{Characteristics of the secondary harmony:}
\smallskip
\pause
\begin{itemize}
	\item Secondary harmony is bounded on a tier, or \pause
	\item Secondary harmony spreads together with the main one, and \pause
	\item Transparent elements for the primary harmony are transparent for the secondary one, and \pause
	\item Blockers for the secondary harmony are presented on a tier of the primary one.
\end{itemize}
\pause
\bigskip
\begin{center}
\textcolor{blue!75!black!50!white}{\textbf{\Large It is always possible to \textit{fit} vowel harmony \\ 
	on one tier!}}
\end{center}
\end{frame}

%------------------------------------------------

\begin{frame}
\frametitle{Why does it matter?}
\begin{itemize}
\item \textcolor{blue!60!black}{What is the algebra of tier alphabets?}
\onslide<2->{
\item If there is more than one tier needed to capture different phenomena in a language, their possible relations are:}
\end{itemize}
%
\begin{figure}[h]
\begin{tikzpicture}
\onslide<3->{\draw[red,fill=red, opacity=.5] (0,0) circle (2em);
\draw[green,fill=green, opacity=.5] (1.6,0) circle (2em);
\node at (0.8,-1.2) {\texttt{Disjoint}};}
\onslide<4->{\node at (0.8,-1.6) {\small \textsc{KiYaka} i.a.};
\node at (0.8,-1.9) {\tiny Hyman (1995, 1998)};}
\onslide<5->{\draw[green,fill=green, opacity=.5] (7,0) circle (2em);
\draw[red,fill=red, opacity=.5] (7,-0.2) circle (1.2em);
\node at (7,-1.2) {\texttt{Embedded}};}
\onslide<6->{\node at (7,-1.6) {\small \textsc{Imdlawn Tshlhiyt}};
\node at (7,-1.9) {\tiny McMullin (2016)};}
\onslide<7->{\draw[green,fill=green, opacity=.5] (3.5,-2.5) circle (2em);
\draw[red,fill=red, opacity=.5] (4.25,-2.5) circle (2em);
\node at (3.875,-3.7) {\texttt{Intersecting}};}
\onslide<8->{\node at (3.875,-4.15) {\large \alert{None?}};}
\end{tikzpicture}
\end{figure}
\end{frame}

%------------------------------------------------

\begin{frame}
\frametitle{On the next episode...}
\begin{itemize}
\item What are double harmonies telling us?
\item What are natural tiers and how do they differ from the unnatural ones?
\item What are the limitations on the tier projection?
\end{itemize}
%
\bigskip\bigskip
%
\begin{tcolorbox}[colback=blue!5!white,colframe=blue!80!red!30!white]
{\centering The more complex the mind, the greater the need for the simplicity of play.}
\begin{flushright}
\emph{James T. Kirk}
\end{flushright}
\end{tcolorbox}
\end{frame}


%------------------------------------------------
% REFERENCES
%------------------------------------------------
\appendix
\section*{References}
\subsection*{References}

{\unnumbered
\begin{frame}%[allowframebreaks]
\frametitle{References}
{ \tiny
\begin{thebibliography}{99}
%
\bibitem{Akinlabi2009} Akinlabi, Akinbiyi (2003)
\newblock Neutral vowels in Lokaa harmony.
\newblock \emph{Canadian Journal of Linguistics} 59(2):197-228.
%
\bibitem{HerbertPoppe63} Herbert, Raymond and Nicholas Poppe (1963)
\newblock Kirghiz manual.
\newblock In \emph{Uralic and Altaic Series} 33. Bloomington: Indiana University publications.
%
\bibitem{Hyman95} Hyman, Larry (1995)
\newblock The case of Yaka.
\newblock In \emph{Studies in African Linguistics} 24:5-30.
%
\bibitem{Hyman98} Hyman, Larry (1998)
\newblock Positional prominence and the `prosodic trough' in Yaka.
\newblock In \emph{Phonology} 15:14-75.
%
\bibitem{Kaun1995} Kaun, Abigail Rhoades (1995)
\newblock The typology of rounding harmony: an optimality theoretic approach.
\newblock PhD thesis, UCLA.
%
%\bibitem{Mahanta08} Mahanta, Shakuntala (2008)
%\newblock Local vs. non-local consonantal intervention in vowel harmony.
%\newblock In \emph{Proceedings of ConSOLE XIV}. Leiden: Leiden University, pp. 165-188.
%
\bibitem{McMullin16} McMullin, Kevin James (2016)
\newblock Tier-based locality in long-distance phonotactics: learnability and typology.
\newblock PhD thesis, University of British Columbia.
%
\end{thebibliography}
}
\end{frame}
}

\section[Kirghiz harmony]{Kirghiz harmony}
\subsection*{Kirghiz harmony}
%------------------------------------------------

\begin{frame}
\frametitle{Secondary harmony: Kirghiz}
\begin{itemize}
	\item Primary harmony is frontness.
	\item Secondary harmony is rounding.
\end{itemize}
%
\pause
%
\begin{example}[Double vowel harmony in \textsc{Kirghiz}]
From \textcolor{gray}{(Kaun 1995)}, dialect described in \textcolor{gray}{(Herbert \& Poppe 1963)}
\begin{itemize}
	\item \textcolor{blue!60!black}{Primary:} all vowels harmonize in frontness.
	\item \textcolor{blue!60!black}{Secondary:}
		\begin{itemize}
			\item In [+front] words, all vowels agree in rounding.
			\item In [--front] words, only adjacent non-high vowels harmonize in rounding, 
				high vowels are blockers.
		\end{itemize}
	%
	\medskip\pause
	%
	\begin{tabular}{lll}
	k\textcolor{green!50!black}{\"{o}}l-d\textcolor{green!50!black}{\"{o}}n & `lake-\textsc{abl}' & *k\"{o}l-den \\
	\textcolor{green!50!black}{\"{u}}y-d\textcolor{green!50!black}{\"{o}}n & `house-\textsc{abl}' & *\"{u}y-den \\
	t\textcolor{green!50!black}{o}k\textcolor{green!50!black}{o}y-d\textcolor{green!50!black}{o}n & `forest-\textsc{abl}' & *tokoy-dan \\
	t\textcolor{green!50!black}{o}rm\textcolor{green!50!black}{u}\v{s}-\textcolor{green!50!black}{t}an & `life-\textsc{abl}' & *tormu\v{s}-ton
	\end{tabular}
\end{itemize}
\end{example}
\end{frame}

%------------------------------------------------

\begin{frame}
\frametitle{Secondary harmony: Kirghiz [cont.]}
\begin{itemize}
	\item $T$ = \{o, \"{o}, e, a, \textrtailz, a, u, \"{u}\}
	\pause\bigskip
	%
	\item \textcolor{blue!60!black}{Primary:} all vowels harmonize in frontness.
	\item No *V$_{\alpha front}$V$_{-\alpha front}$ bigrams \\
		{\small $H_{front}$\,\,= \{*\"{o}o, *ea, *a\"{o}, *u\"{u}, *ae, ...\} }
	\pause\bigskip
	%
	\item \textcolor{blue!60!black}{Secondary:} in [+front] words, all vowels agree in rounding.
	\item No *V$^{+front}_{\alpha round}$V$^{+front}_{-\alpha round}$ bigrams \\
		{\small $H_{round}$\,\,= \{*\"{u}e, *e\"{u}, *\"{o}e, *e\"{o}, *i\"{u}, *\"{u}i, ...\} }
		\pause\bigskip
	%
	\item \textcolor{blue!60!black}{Secondary:} ... and in [--front] words, only adjacent non-high vowels harmonize in rounding.
	\item No *V$^{-hi}_{\alpha round}$V$^{-hi}_{-\alpha round}$ bigrams \\
		{\small $H'_{round}$\,\,= $H_{round}$ + \{*ao, *oa, *uo,  *\textrtailz o\} }
\end{itemize}
\end{frame}

%------------------------------------------------

\begin{frame}
\frametitle{Secondary harmony: Kirghiz [cont.]}
\begin{itemize}
	\item $T$ = \{o, \"{o}, e, a, \textrtailz, a, u, \"{u}\}
	\item {\small $H_{front}$\,\,= \{*\"{o}o, *ea, *a\"{o}, *u\"{u}, *ae, ...\} }
	\item {\small $H_{round}$\,\,= \{*\"{u}e, *e\"{u}, *\"{o}e, *i\"{u}, *\"{u}i, ...\}  + \{*ao, *oa, *uo,  *\textrtailz o\} }
\end{itemize}
\pause
\begin{example}[Double vowel harmony in \textsc{Kirghiz}]
\begin{tabular}{lll}
\hspace{2em} \textcolor{green!50!black}{\"{u}}y-d\textcolor{green!50!black}{\"{o}}n & `house-\textsc{abl}' & *\"{u}y-den
\end{tabular}
\pause
\begin{figure}
\begin{tikzpicture}
\node (0) at (0,0) {\"{u}};
\node (1) at (0.5,-0.09) {y};
\node (2) at (1,0) {d};
\node (3) at (1.5,0) {\"{o}};
\node (4) at (2,-0.05) {n};
%
\node (00) at (0,1) {\"{u}};
\node (03) at (1.5,1) {\"{o}};
%
\foreach \Source/\Target in {%
        0.north/00.south,
	3.north/03.south%
    }
\draw (\Source) to (\Target);
%
\draw[dotted] (-0.25,0.5) to (3,0.5);
\node at (2.5,0.65) {{\tiny harmony}};
%
\node at (0.3,1.5) {$^{\textcolor{green!50!black}{ok}}$\"{u}yd\"{o}n};
\end{tikzpicture}
%
\hspace{1.5em}
%
\begin{tikzpicture}
\node (0) at (0,0) {\"{u}};
\node (1) at (0.5,-0.09) {y};
\node (2) at (1,0) {d};
\node (3) at (1.5,-0.05) {e};
\node (4) at (2,-0.05) {n};
%
\node (00) at (0,1) {\alert{\"{u}}};
\node (03) at (1.5,1) {\alert{e}};
%
\foreach \Source/\Target in {%
        0.north/00.south,
	3.north/03.south%
    }
\draw (\Source) to (\Target);
%
\draw[dotted] (-0.25,0.5) to (3,0.5);
\node at (2.5,0.65) {{\tiny harmony}};
%
\node at (0.3,1.5) {\textcolor{red}{*}\"{u}yden};
\end{tikzpicture}
\end{figure}
\end{example}
\end{frame}

%------------------------------------------------

\begin{frame}
\frametitle{Secondary harmony: Kirghiz [cont.]}
\begin{itemize}
	\item $T$ = \{o, \"{o}, e, a, \textrtailz, a, u, \"{u}\}
	\item {\small $H_{front}$\,\,= \{*\"{o}o, *ea, *a\"{o}, *u\"{u}, *ae, ...\} }
	\item {\small $H_{round}$\,\,= \{*\"{u}e, *e\"{u}, *\"{o}e, *i\"{u}, *\"{u}i, ...\}  + \{*ao, *oa, *uo,  *\textrtailz o\} }
\end{itemize}
%
\begin{example}[Double vowel harmony in \textsc{Kirghiz}]
\begin{tabular}{lll}
\hspace{2em} t\textcolor{green!50!black}{o}rm\textcolor{green!50!black}{u}\v{s}-t\textcolor{green!50!black}{a}n & `life-\textsc{abl}' & *tormu\v{s}-ton
\end{tabular}
\pause
\begin{figure}
\begin{tikzpicture}
\node (0) at (0,0.03) {t};
\node (1) at (0.5,0) {o};
\node (2) at (1,0) {r};
\node (3) at (1.5,0) {m};
\node (4) at (2,0) {u};
\node (5) at (2.5,0.03) {\v{s}};
\node (6) at (3,0.03) {t};
\node (7) at (3.5,0) {a};
\node (8) at (4,0) {n};
%
\node (01) at (0.5,1) {o};
\node (04) at (2,1) {u};
\node (07) at (3.5,1) {a};
%
\foreach \Source/\Target in {%
        1.north/01.south,
        4.north/04.south,
	7.north/07.south%
    }
\draw (\Source) to (\Target);
%
\draw[dotted] (-0.25,0.5) to (4.57,0.5);
\node at (4.1,0.65) {{\tiny harmony}};
%
\node at (0.7,1.5) {$^{\textcolor{green!50!black}{ok}}$tormu\v{s}tan};
\end{tikzpicture}
%
\hspace{0.3em}
%
\begin{tikzpicture}
\node (0) at (0,0.03) {t};
\node (1) at (0.5,0) {o};
\node (2) at (1,0) {r};
\node (3) at (1.5,0) {m};
\node (4) at (2,0) {u};
\node (5) at (2.5,0.03) {\v{s}};
\node (6) at (3,0.03) {t};
\node (7) at (3.5,0) {o};
\node (8) at (4,0) {n};
%
\node (01) at (0.5,1) {o};
\node (04) at (2,1) {\alert{u}};
\node (07) at (3.5,1) {\alert{o}};
%
\foreach \Source/\Target in {%
        1.north/01.south,
        4.north/04.south,
	7.north/07.south%
    }
\draw (\Source) to (\Target);
%
\draw[dotted] (-0.25,0.5) to (4.57,0.5);
\node at (4.1,0.65) {{\tiny harmony}};
%
\node at (0.7,1.5) {\textcolor{red}{*}tormu\v{s}ton};
\end{tikzpicture}
\end{figure}
\end{example}
\end{frame}

%------------------------------------------------

\begin{frame}
\frametitle{Secondary harmony: Kirghiz [cont.]}
\begin{itemize}
	\item \textcolor{blue!60!black}{Primary:} all vowels harmonize in frontness.
	\item \textcolor{blue!60!black}{Secondary:} in [+front] words, all vowels harmonize in rounding, and in [--front] words, only adjacent non-high vowels agree in it, whereas high vowels function as blockers.
	%
	\pause\bigskip
	%
	\item Only one tier is needed:
		\begin{itemize}
			\item Both harmonies operate over the same alphabet.
			\item Secondary harmony is \emph{bounded on the vowel tier} for [--front] words, 
			or \emph{spreads} together with the [+front] feature.
		\end{itemize}
\end{itemize}
\end{frame}

%------------------------------------------------

\section[Quasi-Buryat harmony]{Quasi-Buryat harmony}
\subsection*{Quasi-Buryat harmony}

\begin{frame}
\frametitle{Unattested patterns: Quasi-Buryat}
\begin{itemize}
	\item Secondary harmony involves blocker that was not presented at the tier of the primary one.
\end{itemize}
\pause
%
\begin{example}[Double vowel harmony in \textsc{Quasi-Buryat}]
\begin{itemize}
	\item \textcolor{blue!60!black}{Primary:} all vowels except \emph{i} harmonize in ATR.
	\item \textcolor{blue!60!black}{Secondary:} adjacent non-high vowels harmonize in rounding, high vowel \emph{i} blocks this spreading. \\ \medskip \pause
		\begin{tabular}{lll}
	\textcolor{green!50!black}{\textopeno}r\textcolor{green!50!black}{\textopeno}d & *\textopeno rad & *\textopeno rod \\
	\textcolor{green!50!black}{\textopeno}r\textcolor{red!70!black}{i}l\textcolor{green!50!black}{a}d & *\textopeno ril\textopeno d & *\textopeno riled \\
	t\textcolor{green!50!black}{o}r\textcolor{green!50!black}{o}d & *tored & *torad \\
	t\textcolor{green!50!black}{o}r\textcolor{red!70!black}{i}l\textcolor{green!50!black}{e}d & *torilod & *torilad
\end{tabular}
	\medskip
	\pause
	\item One tier is not enough: \emph{i} \textsc{must not} presented on the ATR harmony tier, and \textsc{must be} presented on the labial assimilation tier.
\end{itemize}
\end{example}
\end{frame}

%------------------------------------------------

\begin{frame}
\frametitle{Unattested patterns: Quasi-Buryat [cont.]}
\begin{itemize}
	\item \textcolor{blue!60!black}{Primary:} all vowels except \emph{i} harmonize in ATR.
	\smallskip\pause
	\item $T_{ATR}$ = \{\textopeno, o, a, e\}
	\item $H_{ATR}$ = \{*\textopeno o, *o\textopeno, *ae, *ea, *\textopeno e, *ao, ...\}
	%
	\bigskip\pause
	%
	\item \textcolor{blue!60!black}{Secondary:} adjacent non-high vowels harmonize in rounding, high vowel \emph{i} blocks this spreading.
	\smallskip\pause
	\item $T_{round}$ = \{i, \textopeno, o, a, e\}
	\item $H_{round}$ = \{*\textopeno a, *a\textopeno, *oe, *eo, *i\textopeno, *io\}
\end{itemize}
\end{frame}

%------------------------------------------------

\begin{frame}
\frametitle{Unattested patterns: Quasi-Buryat [cont.]}
\begin{itemize}
	{\small 
	\item $T_{ATR}$ = \{\textopeno, o, a, e\}
	\item $H_{ATR}$ = \{*\textopeno o, *o\textopeno, *ae, *ea, *\textopeno e, *ao, ...\}
	\item $T_{round}$ = \{i, \textopeno, o, a, e\}
	\item $H_{round}$ = \{*\textopeno a, *a\textopeno, *oe, *eo, *i\textopeno, *io\} }
\end{itemize}
\pause
%
\begin{example}[Double vowel harmony in \textsc{Pseudo-Buryat}]
\vspace{0.8em}
	\begin{tabular}{lll}
		\hspace{2em} t\textcolor{green!50!black}{o}r\textcolor{red!70!black}{i}l\textcolor{green!50!black}{e}d & *torilod & *torilad
	\end{tabular}
\vspace{-0.5em}
%
\pause
%
\begin{figure}
\begin{tikzpicture}
\node (0) at (0,0.027) {t};
\node (1) at (0.5,0) {o};
\node (2) at (1,0) {r};
\node (3) at (1.5,0) {i};
\node (4) at (2,0.04) {l};
\node (5) at (2.5,0) {e};
\node (6) at (3,0.04) {d};
%
\node (01) at (0.5,1) {o};
\node (05) at (2.5,1) {e};
%
\node (001) at (0.5,-1) {o};
\node (003) at (1.5,-1) {i};
\node (005) at (2.5,-1) {e};
%
\foreach \Source/\Target in {%
	001.north/1.south,
	003.north/3.south,
	005.north/5.south,
        1.north/01.south,
	5.north/05.south%
    }
\draw (\Source) to (\Target);
%
\draw[dotted] (-0.25,0.5) to (4.2,0.5);
\node at (3.5,0.65) {{\tiny ATR harmony}};
\draw[dotted] (-0.25,-0.55) to (4.2,-0.55);
\node at (3.5,-0.4) {{\tiny labial harmony}};
%
\node at (0.4,1.5) {$^{\textcolor{green!50!black}{ok}}$toriled};
\end{tikzpicture}
%
\hspace{1em}
%
\begin{tikzpicture}
\node (0) at (0,0.027) {t};
\node (1) at (0.5,0) {o};
\node (2) at (1,0) {r};
\node (3) at (1.5,0) {i};
\node (4) at (2,0.04) {l};
\node (5) at (2.5,0) {o};
\node (6) at (3,0.04) {d};
%
\node (01) at (0.5,1) {o};
\node (05) at (2.5,1) {o};
%
\node (001) at (0.5,-1) {o};
\node (003) at (1.5,-1) {\alert{i}};
\node (005) at (2.5,-1) {\alert{o}};
%
\foreach \Source/\Target in {%
	001.north/1.south,
	003.north/3.south,
	005.north/5.south,
        1.north/01.south,
	5.north/05.south%
    }
\draw (\Source) to (\Target);
%
\draw[dotted] (-0.25,0.5) to (4.2,0.5);
\node at (3.5,0.65) {{\tiny ATR harmony}};
\draw[dotted] (-0.25,-0.55) to (4.2,-0.55);
\node at (3.5,-0.4) {{\tiny labial harmony}};
%
\node at (0.4,1.5) {\textcolor{red}{*}torilod};
\end{tikzpicture}
\end{figure}
\end{example}
\end{frame}

%------------------------------------------------

\begin{frame}
\frametitle{Unattested patterns: Quasi-Buryat [cont.]}
\begin{itemize}
	{\small 
	\item $T_{ATR}$ = \{\textopeno, o, a, e\}
	\item $H_{ATR}$ = \{*\textopeno o, *o\textopeno, *ae, *ea, *\textopeno e, *ao, ...\}
	\item $T_{round}$ = \{i, \textopeno, o, a, e\}
	\item $H_{round}$ = \{*\textopeno a, *a\textopeno, *oe, *eo, *i\textopeno, *io\} }
\end{itemize}
%
\begin{example}[Double vowel harmony in \textsc{Pseudo-Buryat}]
\vspace{0.8em}
	\begin{tabular}{lll}
		\hspace{2em} toriled & *torilod & *torilad
	\end{tabular}
\vspace{-0.5em}
%
\begin{figure}
\begin{tikzpicture}
\node (0) at (0,0.027) {t};
\node (1) at (0.5,0) {o};
\node (2) at (1,0) {r};
\node (3) at (1.5,0) {i};
\node (4) at (2,0.04) {l};
\node (5) at (2.5,0) {e};
\node (6) at (3,0.04) {d};
%
\node (01) at (0.5,1) {o};
\node (05) at (2.5,1) {e};
%
\node (001) at (0.5,-1) {o};
\node (003) at (1.5,-1) {i};
\node (005) at (2.5,-1) {e};
%
\foreach \Source/\Target in {%
	001.north/1.south,
	003.north/3.south,
	005.north/5.south,
        1.north/01.south,
	5.north/05.south%
    }
\draw (\Source) to (\Target);
%
\draw[dotted] (-0.25,0.5) to (4.2,0.5);
\node at (3.5,0.65) {{\tiny ATR harmony}};
\draw[dotted] (-0.25,-0.55) to (4.2,-0.55);
\node at (3.5,-0.4) {{\tiny labial harmony}};
%
\node at (0.4,1.5) {$^{\textcolor{green!50!black}{ok}}$toriled};
\end{tikzpicture}
%
\hspace{1em}
%
\begin{tikzpicture}
\node (0) at (0,0.027) {t};
\node (1) at (0.5,0) {o};
\node (2) at (1,0) {r};
\node (3) at (1.5,0) {i};
\node (4) at (2,0.04) {l};
\node (5) at (2.5,0) {a};
\node (6) at (3,0.04) {d};
%
\node (01) at (0.5,1) {\alert{o}};
\node (05) at (2.5,1) {\alert{a}};
%
\node (001) at (0.5,-1) {o};
\node (003) at (1.5,-1) {i};
\node (005) at (2.5,-1) {a};
%
\foreach \Source/\Target in {%
	001.north/1.south,
	003.north/3.south,
	005.north/5.south,
        1.north/01.south,
	5.north/05.south%
    }
\draw (\Source) to (\Target);
%
\draw[dotted] (-0.25,0.5) to (4.2,0.5);
\node at (3.5,0.65) {{\tiny ATR harmony}};
\draw[dotted] (-0.25,-0.55) to (4.2,-0.55);
\node at (3.5,-0.4) {{\tiny labial harmony}};
%
\node at (0.4,1.5) {\textcolor{red}{*}torilad};
\end{tikzpicture}
\end{figure}
\end{example}
\end{frame}




\end{document}