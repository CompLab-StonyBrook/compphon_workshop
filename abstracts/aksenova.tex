\documentclass[a4paper,12pt]{article}

\usepackage{fixltx2e}
 \renewcommand{\baselinestretch}{0.99}

\usepackage{natbib}
\bibpunct[:]{(}{)}{;}{a}{}{,}
\setlength{\bibsep}{2pt}
\setlength{\bibhang}{0pt}

\usepackage[a4paper, left=2.5cm, right=2.5cm, top=2.5cm, bottom=2.5cm]{geometry}
\usepackage[kerning=true, protrusion=alltext, draft=false]{microtype}

\usepackage[utf8]{inputenc}	% not working with ð and þ; use \eth and \thorn instead
\usepackage[T1]{fontenc}	% scalable EC fonts
\usepackage{mathptmx}
\usepackage{amssymb}
\usepackage{tipx}
\usepackage{tipa}

\usepackage{tikz}
\usetikzlibrary{positioning}
\usepackage{tikz-qtree}

\usepackage[tiny,compact]{titlesec}
\usepackage[small,it]{caption}
\usepackage{mdwlist}
\usepackage{wrapfig}

\setlength{\parindent}{1em}
\setlength{\parskip}{0pt}
\setlength{\parsep}{0pt}
\setlength{\headsep}{0pt}
\setlength{\topskip}{0pt}
\setlength{\topmargin}{0pt}
\setlength{\topsep}{0pt}
\setlength{\partopsep}{0pt}
\setlength{\headheight}{0pt}
\setlength{\headsep}{0pt}
\pagestyle{empty}

\newcommand{\tsb}[1]{\textsubscript{#1}}
\newcommand{\specialcell}[2][l]{%
  \begin{tabular}[#1]{@{}l@{}}#2\end{tabular}}

\begin{document}
\begin{center}
    {\bfseries To Achieve Harmony We Only Need One Tier}\\
    \footnotesize
    \emph{Keywords}: phonology, harmonies, computational linguistics, subregular languages, expressivity
\end{center}
\vskip 0.5em

\paragraph{Overview}
Phonological processes can be either local (local assimilation, word-final devoicing) or non-local (long-distance dissimilation, unbounded harmony).
A non-local process can be described in a local fashion if items involved in that process are projected onto a tier (Heinz\,2011) because it allows to ignore an unbounded amount of irrelevant material in-between them.

I discuss processes that seem to require at least two tiers, namely vowel harmonies with both primary and secondary feature spreadings.
I show that even these complicated systems do not require more than one tier.
This result is important both from typological and computational points of view because limiting harmonies to one tier puts strong restriction on how such systems can look like, hereby bringing possibility of a new typological universal.

\paragraph{Tier-Based Strictly Local phonology}
Heinz (2010) introduced the idea of decreasing complexity of formal languages describing phonological processes from regular (as in Kaplan and Key (1994)) to \emph{subregular} ones.
This approach seems to be a good fit for natural language, since the class of regular languages is too powerful.
The idea of decreasing the power of generator allows to capture existent phonological processes while avoiding the problem of over-prediction.

Two subregular classes encompass the majority of phonotactic dependencies: \emph{Strictly Local} (SL) and \emph{Tier-Based Strictly Local} (TSL).
SL grammars enforce local dependencies by blocking illicit substrings.
TSL grammars capture non-local dependencies by first projecting a tier and then blocking illicit substrings on the projected tier.

For instance, SL grammar can depict obstruent voicing assimilation by simply blocking illicit sequences of consonants where voicing mismatch is found.
As an example of such banned combinations consider *$s$$b$, *$b$$s$, where $b$ is a voiced obstruent and $s$ is a voiceless one.
However, some languages have sibilant harmony, i.e. sibilants located arbitrarily far from each other agree with respect to a certain feature.
For example, in \textsc{Navajo} it is impossible to find sibilants with different anteriority specifications following each other, and there is no upper bound on the amount of intervening vowels and consonants between those two sibilants.
This process is not SL.
No substring of size $k$ will block disagreeing sibilants due to an unbounded amount of other material in-between them.
TSL grammars capture this dependency by enforcing locality among the agreeing units.
If only sibilants are projected on a tier, bigrams such as *$s$\emph{\textesh}, *\emph{\textesh}\hspace{-0.07em}$s$ will block combinations of sibilants disagreeing in anteriority; see the figures below for examples of SL obstruent assimilation and TSL sibilant harmony.

\vspace{-0.2em}
\begin{center}
\begin{tikzpicture}
\node (0) at (0,0) {s};
\node (1) at (0.5,0.03) {b};
\node (2) at (1,-0.02) {e};
\node (3) at (1.5,-0.02) {s};
\node (4) at (2,0.02) {k};
\node (5) at (2.5,-0.03) {a};
%
\node (00) at (0,1) {s};
\node (03) at (1.5,1) {s};
%
\foreach \Source/\Target in {%
        0.north/00.south,
	3.north/03.south%
    }
\draw (\Source) to (\Target);
%
\draw[dotted] (-1.4,0.5) to (3.45,0.5);
\node at (3,0.65) {{\tiny sibilants}};
%
\node at (-1,0.05) {\textbf{SL}};
\node at (-0.88,1) {\textbf{TSL}};
%
\node at (0.2,1.5) {*sbeska};
%
\node at (0,-0.2) {};
\draw [dashed] (-0.15,0.25) -- (0.7,0.25) -- (0.7,-0.2) -- (-0.15,-0.2) -- (-0.15,0.25);
\end{tikzpicture}
%
%
%
\begin{tikzpicture}
\node (0) at (0,-0.02) {s};
\node (1) at (0.5,-0.06) {p};
\node (2) at (1,-0.02) {e};
\node (3) at (1.5,0) {\textesh};
\node (4) at (2,0.02) {k};
\node (5) at (2.5,-0.03) {a};
%
\node (00) at (0,1) {s};
\node (03) at (1.5,1) {\textesh};
%
\foreach \Source/\Target in {%
        0.north/00.south,
	3.north/03.south%
    }
\draw (\Source) to (\Target);
%
\draw[dotted] (-1.4,0.5) to (3.45,0.5);
\node at (3,0.65) {{\tiny sibilants}};
%
\node at (-1,0.05) {\textbf{SL}};
\node at (-0.88,1) {\textbf{TSL}};
%
\node at (0.2,1.5) {*spe\textesh ka};
%
\draw [dashed] (-0.15,0.76) -- (1.65,0.76) -- (1.65,1.25) -- (-0.15,1.25) -- (-0.15,0.76);
\end{tikzpicture}
%
%
%
\begin{tikzpicture}
\node (0) at (0,0) {\textesh};
\node (1) at (0.5,-0.06) {p};
\node (2) at (1,-0.02) {e};
\node (3) at (1.5,0) {\textesh};
\node (4) at (2,0.02) {k};
\node (5) at (2.5,-0.03) {a};
%
\node (00) at (0,1) {\textesh};
\node (03) at (1.5,1) {\textesh};
%
\foreach \Source/\Target in {%
        0.north/00.south,
	3.north/03.south%
    }
\draw (\Source) to (\Target);
%
\draw[dotted] (-1.4,0.5) to (3.45,0.5);
\node at (3,0.65) {{\tiny sibilants}};
%
\node at (-1,0.05) {\textbf{SL}};
\node at (-0.88,1) {\textbf{TSL}};
%
\node at (0.2,1.5) {$^{ok}$\textesh pe\textesh ka};
\end{tikzpicture}
\end{center}
\vspace{-0.2em}

The same technique can be used to enforce vowel harmonies.
Projecting only relevant vowels on a tier establishes locality relations among them, so illicit sequences can be ruled out.


\paragraph{Attested harmony patterns}
There are many languages with spreading of more than one feature among vowels, i.e.\ with primary and secondary harmonies.
This could be handled by having one tier for each harmony, but it would increase expressivity and such system would be hard to learn: inferring two tiers of dependencies is harder then just one.
I argue that these patterns do not require projection of a separate tier for the secondary feature spreading.
One tier is always enough to capture them both.
Here I describe a language with an ostensibly complicated harmony system, and show how this pattern can be analyzed by projecting just a single tier.
\textsc{Buryat} is discussed as an illustrative example, but similar systems can be found in numerous other languages (\textsc{Yakut}, \textsc{Kirghiz}, \textsc{Khalkha}, etc.), cf. Kaun (1995).
\smallskip

\noindent{\textsc{Buryat} \quad 
The primary harmony is in ATR, therefore vowels within a word can be either lax (1,2) or tense (3,4).
As a secondary harmony, adjacent non-high vowels agree in rounding (1,3) unless a high vowel intervenes (2,4).
The rounding of non-high vowels is phonemic, i.e.\ rounded non-high vowels are limited to a stem-initial syllable; in other positions they can only be licensed by the labial spreading (Comrie 1981).
\smallskip

\noindent%
\begin{tabular}{lll}
\hspace{-0.5em}(1)\, \textopeno r-\textopeno:d \hspace{-0.8em} & `enter-\textsc{perf}' \hspace{-0.7em} & *\textopeno r-a:d \\
\hspace{-0.5em}(2)\, \textopeno r-\textupsilon:l-a:d \hspace{-0.8em} & `enter-\textsc{caus-perf}' \hspace{-0.7em} & *\textopeno r-\textupsilon:l-\textopeno:d \\
\hspace{-0.5em}(3)\, to:r-o:d \hspace{-0.8em} & `wander-\textsc{perf}' \hspace{-0.7em} & *to:r-e:d \\
\hspace{-0.5em}(4)\, to:r-u:l-e:d \hspace{-0.8em} & `wander-\textsc{caus-perf}' \hspace{-0.7em} & *to:r-u:l-o:d
\end{tabular}\hspace{1.3em}%
%
\textcolor{gray!80!white}{
\begin{tabular}{|rllllll|}
			\hline
			 & /e/ & /a/ & /u/ & /\textupsilon/ & /\textopeno/ & /o/ \\
			ATR & + & -- & + & -- & -- & + \\
			round & -- & -- & + & + & + & + \\
			high & -- & -- & + & + & -- & -- \\ \hline
\end{tabular}}

\medskip

This data pattern can be captured with a single tier, which only contains vowels.
First, to capture ATR harmony, adjacent vowels differently specified for [tense] are blocked:
$H_{[ATR]}$\,\,= \{*\textopeno o, *o\textopeno, *\textupsilon u, *u\textupsilon, *ae, *ea, *\textupsilon o, *o\textupsilon, *\textupsilon e, *e\textupsilon, *\textopeno u, *u\textopeno, *\textopeno e, *e\textopeno, *oa, *ao, *au, *ua\}.
Second, to enforce labial assimilation among adjacent non-high vowels and block occurrence of rounded non-high vowels after high ones, $H_{[round]}$\,\,= \{*\textopeno a, *a\textopeno, *eo, *oe, *\textupsilon\textopeno, *uo\} is needed.

\vspace{-0.2em}
\hspace{1.3em}
%
\begin{tikzpicture}
\node (0) at (0,0.027) {t};
\node (1) at (0.5,0) {o:};
\node (2) at (1,0) {r};
\node (3) at (1.5,0) {o:};
\node (4) at (2,0.04) {d};
%
\node (01) at (0.5,1) {o:};
\node (03) at (1.5,1) {o:};
%
\foreach \Source/\Target in {%
        1.north/01.south,
	3.north/03.south%
    }
\draw (\Source) to (\Target);
%
\draw[dotted] (-0.25,0.5) to (3,0.5);
\node at (2.5,0.65) {{\tiny harmony}};
%
\node at (0.3,1.5) {$^{ok}$to:ro:d};
\end{tikzpicture}
%
\hspace{1em}
%
\begin{tikzpicture}
\node (0) at (0,0.027) {t};
\node (1) at (0.5,0) {o:};
\node (2) at (1,0) {r};
\node (3) at (1.5,0) {u:};
\node (4) at (2,0.04) {l};
\node (5) at (2.5,0) {e:};
\node (6) at (3,0.04) {d};
%
\node (01) at (0.5,1) {o:};
\node (03) at (1.5,1) {u:};
\node (05) at (2.5,1) {e:};
%
\foreach \Source/\Target in {%
        1.north/01.south,
        3.north/03.south,
	5.north/05.south%
    }
\draw (\Source) to (\Target);
%
\draw[dotted] (-0.25,0.5) to (4,0.5);
\node at (3.5,0.65) {{\tiny harmony}};
%
\node at (0.4,1.5) {$^{ok}$to:ru:le:d};
\end{tikzpicture}
%
\hspace{1em}
%
\begin{tikzpicture}
\node (0) at (0,0.027) {t};
\node (1) at (0.5,0) {o:};
\node (2) at (1,0) {r};
\node (3) at (1.5,0) {u:};
\node (4) at (2,0.04) {l};
\node (5) at (2.5,0) {o:};
\node (6) at (3,0.04) {d};
%
\node (01) at (0.5,1) {o:};
\node (03) at (1.5,1) {u:};
\node (05) at (2.5,1) {o:};
%
\foreach \Source/\Target in {%
        1.north/01.south,
        3.north/03.south,
	5.north/05.south%
    }
\draw (\Source) to (\Target);
%
\draw[dotted] (-0.25,0.5) to (4,0.5);
\node at (3.5,0.65) {{\tiny harmony}};
\draw [dashed] (1.2,0.77) -- (2.7,0.77) -- (2.7,1.2) -- (1.2,1.2) -- (1.2,0.77);
%
\node at (0.4,1.5) {*to:ru:lo:d};
\end{tikzpicture}\vspace{-0.2em}

In \textsc{Buryat}, as well as in other languages with double vowel harmonies (\textsc{Yakut, Kirgiz, Khalkha}, etc.), the secondary harmony is always locally bound on the tier of vowels.
Local processes can be captured locally, so there no need in a distinct tier for the secondary harmony.


\paragraph{Typologically absent harmony}
Suppose that there is a language where the secondary harmony is also unbounded and hence not local over the vowel tier.
\smallskip

\noindent{\textsc{Pseudo-Buryat} \quad 
The primary ATR harmony is the same as before, so we can use the same set $H_{[ATR]}$ over the vowel tier.
But labial assimilation now occurs among all non-high vowels rather than just adjacent ones.
Capturing the rounding assimilation with $H'_{[round]}$ = \{*a\textopeno, *\textopeno a, *eo, *oe\} will not result in the desired pattern because there can be an unbounded amount of high vowels in-between the two agreeing non-high ones, as opposed to \textsc{Buryat}.
A separate tier of non-high vowels has to be projected in order to achieve locality among all non-high vowels.

\vspace{-0.2em}
\hspace{4em}
%
\begin{tikzpicture}
\node (0) at (0,0.027) {t};
\node (1) at (0.5,0) {o};
\node (2) at (1,0) {r};
\node (3) at (1.5,0) {u};
\node (4) at (2,0.04) {l};
\node (5) at (2.5,0) {o};
\node (6) at (3,0.04) {d};
%
\node (01) at (0.5,1) {o};
\node (03) at (1.5,1) {u};
\node (05) at (2.5,1) {o};
%
\node (001) at (0.5,-1) {o};
\node (005) at (2.5,-1) {o};
%
\foreach \Source/\Target in {%
	001.north/1.south,
	005.north/5.south,
        1.north/01.south,
        3.north/03.south,
	5.north/05.south%
    }
\draw (\Source) to (\Target);
%
\draw[dotted] (-0.25,0.5) to (4.2,0.5);
\node at (3.5,0.65) {{\tiny ATR harmony}};
\draw[dotted] (-0.25,-0.55) to (4.2,-0.55);
\node at (3.5,-0.4) {{\tiny labial harmony}};
%
\node at (0.4,1.5) {$^{ok}$torulod};
\end{tikzpicture}
%
\hspace{5em}
%
\begin{tikzpicture}
\node (0) at (0,0.027) {t};
\node (1) at (0.5,0) {o};
\node (2) at (1,0) {r};
\node (3) at (1.5,0) {u};
\node (4) at (2,0.04) {l};
\node (5) at (2.5,0) {e};
\node (6) at (3,0.04) {d};
%
\node (01) at (0.5,1) {o};
\node (03) at (1.5,1) {u};
\node (05) at (2.5,1) {e};
%
\node (001) at (0.5,-1) {o};
\node (005) at (2.5,-1) {e};
%
\foreach \Source/\Target in {%
	001.north/1.south,
	005.north/5.south,
        1.north/01.south,
        3.north/03.south,
	5.north/05.south%
    }
\draw (\Source) to (\Target);
%
\draw[dotted] (-0.25,0.5) to (4.2,0.5);
\node at (3.5,0.65) {{\tiny ATR harmony}};
\draw[dotted] (-0.25,-0.55) to (4.2,-0.55);
\node at (3.5,-0.4) {{\tiny labial harmony}};
\draw [dashed] (0.28,-0.8) -- (2.7,-0.8) -- (2.7,-1.2) -- (0.28,-1.2) -- (0.28,-0.8);
%
\node at (0.4,1.5) {*toruled};
\end{tikzpicture}
\vspace{-0.2em}

This category of patterns is unattested to the best of my knowledge.
In a language with both primary and secondary vowel harmonies, the secondary one is always bound on a vowel tier.
This suggests a novel typological universal: \emph{all harmony patterns are TSL with a single tier}.


\paragraph{Discussion}
I have argued that even for complex harmony systems with spreading of more than one feature, TSL grammar with a single tier is enough to capture the desired pattern.
Harmony patterns where secondary harmony is also unbounded among certain subset of vowels are a conceivable option given no a priori assumptions.
However, natural languages establish limitations on the secondary spreading, rendering it locally bounded over a vowel tier.
Such universal limitation is quite puzzling from the language-theoretical point of view, but finds in explanation from a TSL perspective to phonology.
This shows that this approach can unearth and explain previously unnoticed properties of natural languages.


\medskip
\scriptsize
\newcommand{\bibseparator}{\tikz\draw[draw=gray!80,fill=gray!80] (0,0) circle (.5ex);\,}
\noindent%
\textbf{Comrie}, B. 1981.
\newblock \emph{The Languages of the Soviet Union}.
\newblock Foreign Language Study, CUP Archive.
$\bibseparator$
\textbf{Heinz}, J. 2010.
\newblock Learning Long-Distance Phonotactics.
\newblock {\em LI\/} 41(4):623--661.
$\bibseparator$
\textbf{Heinz}, J., C. Rawal, and H.~G. Tanner. 2011.
\newblock Tier-based strictly local constraints in phonology.
\newblock In {\em ACL 49\/}, 58--64.
$\bibseparator$
\textbf{Kaplan}, R.~M., and M. Kay. 1994.
\newblock Regular models of phonological rule systems.
\newblock {\em CL\/} 20:331--378.
$\bibseparator$
\textbf{Kaun}, A.~R. 1995.
\newblock The typology of rounding harmony: an optimality theoretic approach.
\newblock PhD thesis, UCLA.
\end{document}
